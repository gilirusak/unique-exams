Let S equal the number of Stanford students who have a full-time job and B equal the number of Berkeley undergraduates who have full-time jobs.  We want to compute P(S + B = 9), which is trivially a convolution, but more easier expressed as a sum of two mutually exclusive events.

\begin{equation*}
\begin{split}
P(S + B = 9) &= P(S = 5, B = 4) + P(S = 4, B = 5) \\
             &= P(S = 5) \, P(B = 4) + P(S = 4)\cdot P(B = 5) \\
             &= \binom{5}{4} (0.\STANFORDRATE)^{4} (0.\STANFORDRATECOMP) \binom{5}{5} (0.\BERKELEYRATE)^{5} + \binom{5}{5} (0.\STANFORDRATE)^{5} \binom{5}{4} (0.\BERKELEYRATE)^{4} (0.09) \\
             &= \binom{5}{4} \binom{5}{5} (0.\STANFORDRATE)^{4} (0.\STANFORDRATECOMP) (0.91)^{5} + \binom{5}{5} \binom{5}{4} (0.\STANFORDRATE)^{5} (0.\BERKELEYRATE)^{4} (0.09) \\
             &= 5 \cdot (0.\STANFORDRATE)^{4} (0.\STANFORDRATECOMP) (0.\BERKELEYRATE{})^{5} + 5 \cdot (0.\STANFORDRATE)^{5} (0.\BERKELEYRATE{})^{4} (0.\BERKELEYRATECOMP) \\
             &= \BERNOULLISOLNA{}
\end{split}
\end{equation*}