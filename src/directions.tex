\vspace*{-2em}
\section*{Take-Home Quiz information}
Each quiz will be a 47-hour open-book, open-note exam. We have designed this quiz to approximate about 1-3 hours of active work (\textit{before} typesetting). 
\vspace*{-1em}
\begin{itemize}[itemsep=0pt]
\item You can submit multiple times; we will only grade the last submission you submit before 1:00pm (Pacific time) on Friday, October 30$^\text{th}$. No exam submissions will be accepted late. When uploading, please assign pages to each question. Failure to do so will result in a 2-point deduction. \textit{\textbf{Please double-check that you submit the right file.}}
\item You should upload your submission as a PDF to Gradescope. We provide a LaTeX template if you find it useful, but we will accept any legible submission.  You may also find the CS109 Probability LaTeX reference useful: {\small \url{https://www.overleaf.com/read/wyhtzmdsfwkb}}
\item Course staff assistance will be limited to clarifying questions of the kind that might be allowed on a traditional, in-person exam. If you have questions during the exam, please ask them as private posts via our discussion forum. We will not have any office hours for answering quiz questions during the quiz.
\item \textbf{For each problem, briefly explain/justify how you obtained your answer} at a level such that a future CS109 student would be able to understand how to solve the problem. It is fine for your answers to be a well-defined mathematical expression including summations (but not integrals), products, factorials, exponentials, and combinations, unless the question \textit{specifically} asks for a numeric quantity or closed form. Where numeric answers are required, fractions are fine.
\end{itemize} 

\vspace*{-1em}
\section*{Honor Code Guidelines for Take-Home Quizzes}
\textbf{\textit{This exam must be completed individually.}} It is a violation of the Stanford Honor Code to communicate with any other humans about this exam (other than CS109 course staff), to solicit solutions to this exam, or to share your solutions with others.

The take-home exams are open-book: open lecture notes, handouts, textbooks, course lecture videos, and internet searches for conceptual information (e.g., Wikipedia). Consultation of other humans in any form or medium (e.g., communicating with classmates, asking questions on sites like Chegg or Stack Overflow) is prohibited. All work done with the assistance of any external material in any way (other than provided CS109 course materials) must include citation (e.g., ``Referred to Wikipedia page on $X$ for Question 2.''). Copying solutions is unacceptable, even with citation. If by chance you encounter solutions to the problem, navigate away from that page before you feel tempted to copy.

If you become aware of any Honor Code violations by any student in the class, your commitments under the Stanford Honor Code obligate you to inform course staff. \textit{\textbf{Please remember that there is no reason to violate your conscience to complete a take-home exam in CS109.}}

\vspace*{1em}

I acknowledge and accept the letter and spirit of the Honor Code:

\vspace*{1.5em}
Name (typed or written): \blank{} % if in LaTeX, write your name in the curly braces

\setlist{nolistsep}